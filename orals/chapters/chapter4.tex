
%*****************************************************************************************
%   Chapter 4
%*****************************************************************************************

%=========================================================================================
%   Beta Ensembles
%=========================================================================================
\section{$\b$-ensembles}
%=========================================================================================

\begin{frame} \frametitle{$\b$-ensembles}

\begin{alertblock}{\deftitle{$\b$-ensembles}}
A (Hermite) $\beta$-ensemble is an ensemble of random matrices parameterized by $\beta$, which determines the joint eigenvalue p.d.f that characterizes it.
So, given an observed set of eigenvalues $\L = (\seqo[N]{\lambda})$. Then, the joint p.d.f. of $\L$ is as follows:
\begin{align*}
\jepdf = C_\beta \prod_{i < j} |\lambda_i - \lambda_j|^\beta e^{-\frac{1}{2} \sum_{i=1}^N \lambda_i^2}
%f_\beta(\Lambda) = c_\beta \prod_{i < j} |\lambda_i - \lambda_j|^\beta \exp({-\frac{1}{2} \sum_{i=1}^N \lambda_i^2})
\end{align*}
where the normalization constant $C_\beta$ is given by:
\begin{align*}
C_{\beta} = (2\pi)^{-n/2} \prod_{j = 1}^n \frac{\Gamma(1 + \frac{\b}{2})}{\Gamma(1 + \frac{\b}{2}j)}
\end{align*}
\end{alertblock}

\end{frame}
% %=========================================================================================
\begin{frame} \frametitle{$\b$-ensembles: Breaking it Down}
  \begin{align*}
  \jepdf = C_\beta \prod_{i < j} |\lambda_i - \lambda_j|^\beta e^{-\frac{1}{2} \sum_{i=1}^N \lambda_i^2}
  %f_\beta(\Lambda) = c_\beta \prod_{i < j} |\lambda_i - \lambda_j|^\beta \exp({-\frac{1}{2} \sum_{i=1}^N \lambda_i^2})
  \end{align*}
\end{frame}
% %=========================================================================================
\begin{frame} \frametitle{$\b$-ensembles: Breaking it Down}
  Altogether, here is what we can say about the $\beta$-ensemble joint eigenvalue p.d.f just from observing the terms.
    \begin{enumerate}
      %\item When $\l_i \in \L$ is large, then $\P(\L)$ is small.
      %\item When $\d(\l_i, \l_j) \for \l_i,\l_j \in \L$ is small, then $\P(\L)$ is small.
      \item When $\l_i$ is large, then $\P(\L)$ is small.
      \item When $\d(\l_i, \l_j)$ is small, then $\P(\L)$ is small.
    \end{enumerate}
\end{frame}
%=========================================================================================
%=========================================================================================
\begin{frame} \frametitle{Charged Particle Model}

  Alongside these great algebraic properies, the $\b$-ensembles also have a intepretation as a physical model.
  This is because as mentioned previously, these ensembles show up frequently in statistical physics.
  So, we will cover one physical model that the $\b$-ensemble represents.

  Suppose that $P \sim \H(\beta)$ is an $N \times N$ matrix. Then, the eigenvalues of $P$ have a representation as a model of charged point particles.

\end{frame}
%=========================================================================================
\begin{frame} \frametitle{Charged Particle Model}
  \blocktitle{Low Replsuion, High Temperature} As $\b \to 0$, the temperature of the system $T \to \infty$.
  At these values, the model starts to behave like an ideal gas.
   This returns a \textbf{fully stochastic} model of particles trying to align themselves along the field potential.

  \blocktitle{High Repulsion, Low Temperature} As $\b \to \infty$, the temperature of the system $T \to 0$.
  At these values, the model loses its stochastic properties and starts to become \textbf{deterministic}. This returns a fully deterministic*
  model of particles that align themselves equidistantly.

\end{frame}
%=========================================================================================
\begin{frame} \frametitle{Wigner's Surmise}
  %FFFFFFFFFFFFFFFFFFFFFFFFFFFFFFFFFFFFFFFFFFFFFFFFFFFFFFFFFFFFFFFFFFFFFFFFFFF
  \plotwrapperNC{h}{0.3}{../graphics/chap3/3-3_wigner_beta_extended}%{Wigner's Surmise for Non-Standard Beta Ensembles}
  %FFFFFFFFFFFFFFFFFFFFFFFFFFFFFFFFFFFFFFFFFFFFFFFFFFFFFFFFFFFFFFFFFFFFFFFFFFF
\end{frame}
%=========================================================================================
\begin{frame} \frametitle{Conclusion}
  In the end, we simulated and verified some results, found a few interesting patterns and phenomena.
\end{frame}


%=========================================================================================
%=========================================================================================
