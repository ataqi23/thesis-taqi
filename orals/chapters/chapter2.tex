
%*****************************************************************************************
%   Chapter 2
%*****************************************************************************************

%=========================================================================================
%   Spectra
%=========================================================================================
\section{Spectra}
%=========================================================================================
\begin{frame} \frametitle{Spectra}

\begin{alertblock}{\deftitle{Spectrum}}
Suppose $P \in \F^{N \times N}$ is a square matrix of size $N$ over $\F$. Then, the (eigenvalue) spectrum of $P$ is defined as the multiset of its eigenvalues and it is denoted
$$\sigma(P) = \{\lambda_i \in \Cc \mid \text{char}_P(\lambda_i) = 0 \}_{i=1}^N$$
Note that it is important to specify that a spectrum is a multiset and not just a set; eigenvalues could be repeated due to algebraic multiplicity and we opt to always have $N$ eigenvalues.
\end{alertblock}

\end{frame}
%=========================================================================================
\begin{frame} \frametitle{Ensemble Spectrum}

\begin{alertblock}{\deftitle{Ensemble Spectrum}}
Let $\Ens \sim \D$ be an ensemble of matrices $P_i \in \F^{n \times n}$. To take the spectrum of $\Ens$, simply take the union of the spectra of each of its matrices.
In other words, if $\Ens = \{P_i \sim \mathcal{D}\}_{i = 1}^K$, then we denote the spectrum of the ensemble
$$\sigma(\Ens) = \bigcup_{i=1}^K \sigma(P_i)$$
\end{alertblock}

\end{frame}

% %=========================================================================================
% \begin{frame}
% \frametitle{Empty Slide}
% \end{frame}
% %=========================================================================================
% \begin{frame}
% \frametitle{Empty Slide}
% \end{frame}
% %=========================================================================================
% \begin{frame}
% \frametitle{Empty Slide}
% \end{frame}


%=========================================================================================
%=========================================================================================


%=========================================================================================
%   Spectrum Analysis
%=========================================================================================
\section{Spectrum Analysis}
%=========================================================================================

\begin{frame} \frametitle{Order Statistics}

One method of analyzing the spectrum of a matrix is to consider the framework of using order statistics.

\begin{examples}
Let's say you roll a die 3 times and obtain 5,6, and 2.
Then, your order statistics take the form $X_1 = 6$, $X_2 = 5$, and $X_3 = 2$.
\end{examples}

\end{frame}

%=========================================================================================

\begin{frame} \frametitle{Order Statistics}

That being said, consider the two order schema that we will use.

\begin{enumerate}
  \item The \textbf{sign}-ordering scheme; works on $\R$. \begin{align*}
  \sigma_S(P) = \{\lambda_j : \lambda_1 \geq \lambda_2 \geq \dots \geq \lambda_N\}_{j = 1}^N
  \end{align*}
  \item  The \textbf{norm}-ordering scheme; works on $\R$ and $\C$. \begin{align*}
  \sigma_N(P) = \{\lambda_j : |\lambda_1| \geq |\lambda_2| \geq \dots \geq |\lambda_N|\}_{j = 1}^N
  \end{align*}
\end{enumerate}

\end{frame}

%=========================================================================================

\begin{frame} \frametitle{Order Statistics}

Consider the following example:

%FFFFFFFFFFFFFFFFFFFFFFFFFFFFFFFFFFFFFFFFFFFFFFFFFFFFFFFFFFFFFFFFFFFFFFFFFFF
\plotwrapper{h}{0.35}{../graphics/chap2/2-2-1_orderscheme}
{Spectrum of an Ensemble Using Two Different Ordering Schemes}
%FFFFFFFFFFFFFFFFFFFFFFFFFFFFFFFFFFFFFFFFFFFFFFFFFFFFFFFFFFFFFFFFFFFFFFFFFFF

\end{frame}

%=========================================================================================

\begin{frame} \frametitle{Order Statistics}

\blocktitle{Expectation} $\E(\lambda_{i} \mid i)$ One useful summary statistic to consider when analyzing a spectrum is the expected norm or value (which we will just call quantity hereinafter) of the eigenvalue at the $i^{th}$ rank.

\medskip

\blocktitle{Variance} $\Var(\lambda_{i} \mid i)$ Similarly, the variance of the eigenvalue quantity at a given order $i$ can tell us a lot about an ensemble.

\end{frame}

%=========================================================================================





%=========================================================================================
%   Symmetric & Hermitian Matrices
%=========================================================================================
\section{Symmetric $\&$ Hermitian Matrices}
%=========================================================================================

\begin{frame} \frametitle{Symmetric $\&$ Hermitian Matrices}

Symmetric $\&$ Hermitian matrices are an important class of matrices in Linear Algebra.
Suppose we have a matrix $P$.

\begin{enumerate}
  \item P Symmetric $\iff$ $p_{ij} = p_{ji}$
  \item P Hermitian $\iff$ $p_{ij} = \bar{p_{ji}}$
\end{enumerate}

%FFFFFFFFFFFFFFFFFFFFFFFFFFFFFFFFFFFFFFFFFFFFFFFFFFFFFFFFFFFFFFFFFFFFFFFFFFF
\plotwrapperNC{h}{0.35}{graphics/symmetric}
%FFFFFFFFFFFFFFFFFFFFFFFFFFFFFFFFFFFFFFFFFFFFFFFFFFFFFFFFFFFFFFFFFFFFFFFFFFF

\end{frame}
%=========================================================================================

\begin{frame} \frametitle{Symmetric $\&$ Hermitian Matrices}

So, why do we care?

\begin{alertblock}{\textbf{Theorem}}
Suppose $P$ is Symmetric/Hermitian. Then, $P$ has a set of real eigenvalues. That is,
$$P \textbf{ Symmetric/Hermitian} \implies \forall \lambda_i \in \sigma(P) \mid \l_i \in \R$$
\end{alertblock}

\end{frame}
%=========================================================================================

\begin{frame} \frametitle{Wigner's Semicircle Distribution}

\begin{alertblock}{\deftitle{Wigner's Semicircle Distribution}}
If a random variable $X$ is semicircle distributed with radius $R \in \R^+$, then we say $X \sim \text{SC}(R)$. $X$ has the following probability density function:
$$\Prb(X = x) = \frac{2}{\pi R^2} \sqrt{R^2 - x^2} \for x \in [-R, R]$$
\end{alertblock}

\end{frame}
%=========================================================================================

\begin{frame} \frametitle{Wigner's Semicircle Distribution}

  %FFFFFFFFFFFFFFFFFFFFFFFFFFFFFFFFFFFFFFFFFFFFFFFFFFFFFFFFFFFFFFFFFFFFFFFFFFF
  \plotwrapperNC{h}{0.4}{../graphics/chap2/2-3-2_semicircle}%{Eigenvalues of a Symmetric Matrix displaying the Semicircle Distribution}
  %FFFFFFFFFFFFFFFFFFFFFFFFFFFFFFFFFFFFFFFFFFFFFFFFFFFFFFFFFFFFFFFFFFFFFFFFFFF

\end{frame}
%=========================================================================================

\begin{frame} \frametitle{Example: Real Normal Symmetric Matrices}

  In this section, we will be carefully analyzing an ensemble of symmetric matrices to showcase
  the special properties of symmetric and hermitian matrices.
  Namely, we will be considering an ensemble $\Ens \sim \Normal(0,1)^\dagger$ of $15 \times 15$ matrices over $\R$.

\end{frame}
%=========================================================================================

\begin{frame} \frametitle{Example: Real Normal Symmetric Matrices}

  In this section, we will be carefully analyzing an ensemble of symmetric matrices to showcase
  the special properties of symmetric and hermitian matrices.
  Namely, we will be considering an ensemble $\Ens \sim \Normal(0,1)^\dagger$ of $15 \times 15$ matrices over $\R$.

\end{frame}
%=========================================================================================

\begin{frame} \frametitle{Example: Real Normal Symmetric Matrices}

  \plotwrapperNC{h}{0.2}{../graphics/chap2_order/sign_Re}%{Largest eigenvalues distribution}
  \plotwrapperNC{h}{0.2}{../graphics/chap2_order/sign_Re_var}%{Largest eigenvalues distribution}

\end{frame}
%=========================================================================================

\begin{frame} \frametitle{Example: Real Normal Symmetric Matrices}

  \plotwrapperNC{h}{0.2}{../graphics/chap2_order/norm_Re}%{Largest eigenvalues distribution}
  \plotwrapperNC{h}{0.2}{../graphics/chap2_order/norm_Re_var}%{Largest eigenvalues distribution}

\end{frame}
%=========================================================================================

\begin{frame} \frametitle{Example: Real Normal Symmetric Matrices}

  \plotwrapperNC{h}{0.2}{../graphics/chap2_order/norm_Norm}%{Largest eigenvalues distribution}
  \plotwrapperNC{h}{0.2}{../graphics/chap2_order/norm_Norm_var}%{Largest eigenvalues distribution}

\end{frame}
%=========================================================================================

%=========================================================================================
%=========================================================================================






%=========================================================================================
%   A Survey Of Spectra
%=========================================================================================
\section{A Survey Of Spectra}
%=========================================================================================

\begin{frame} \frametitle{Uniform Ensemble Spectra}

\plotwrapper{h}{0.075}{../graphics/chap2/2-4_unif01_spec}{Spectrum of a Uniform(0,1) Matrix ensemble}

\end{frame}

%=========================================================================================

\begin{frame} \frametitle{Stochastic Matrix Ensemble Spectra}

  %FFFFFFFFFFFFFFFFFFFFFFFFFFFFFFFFFFFFFFFFFFFFFFFFFFFFFFFFFFFFFFFFFFFFFFFFFFF
  \plotwrapper{h}{0.1}{../graphics/chap2/2-4_stoch_spec}{Spectrum of a Stochastic Matrix ensemble}
  %FFFFFFFFFFFFFFFFFFFFFFFFFFFFFFFFFFFFFFFFFFFFFFFFFFFFFFFFFFFFFFFFFFFFFFFFFFF

  %FFFFFFFFFFFFFFFFFFFFFFFFFFFFFFFFFFFFFFFFFFFFFFFFFFFFFFFFFFFFFFFFFFFFFFFFFFF
  \plotwrapper{h}{0.3}{../graphics/chap2/2-4_symmstoch_spec}{Spectrum of a Symmetric Stochastic Matrix ensemble}
  %FFFFFFFFFFFFFFFFFFFFFFFFFFFFFFFFFFFFFFFFFFFFFFFFFFFFFFFFFFFFFFFFFFFFFFFFFFF

\end{frame}

%=========================================================================================

\begin{frame} \frametitle{Erdos-Renyi Ensemble Spectra}

  %FFFFFFFFFFFFFFFFFFFFFFFFFFFFFFFFFFFFFFFFFFFFFFFFFFFFFFFFFFFFFFFFFFFFFFFFFFF
  \plotwrapperNC{h}{0.3}{../graphics/chap2/2-4_erdos_lam2}%{Second Largest Pair of Eigenvalues of an Erdos-p ensemble}
  %FFFFFFFFFFFFFFFFFFFFFFFFFFFFFFFFFFFFFFFFFFFFFFFFFFFFFFFFFFFFFFFFFFFFFFFFFFF

\end{frame}

%=========================================================================================








%=========================================================================================
%=========================================================================================
