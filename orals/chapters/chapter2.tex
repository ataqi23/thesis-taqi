
%*****************************************************************************************
%   Chapter 2
%*****************************************************************************************

%=========================================================================================
%   Spectra
%=========================================================================================
\section{Spectra}
%=========================================================================================
\begin{frame} \frametitle{Spectra}

\begin{alertblock}{\deftitle{Spectrum}}
Suppose $P \in \F^{N \times N}$ is a square matrix of size $N$ over $\F$. Then, the (eigenvalue) spectrum of $P$ is defined as the multiset of its eigenvalues and it is denoted
$$\sigma(P) = \{\lambda_i \in \Cc \mid \text{char}_P(\lambda_i) = 0 \}_{i=1}^N$$
Note that it is important to specify that a spectrum is a multiset and not just a set; eigenvalues could be repeated due to algebraic multiplicity and we opt to always have $N$ eigenvalues.
\end{alertblock}

\end{frame}
%=========================================================================================
\begin{frame} \frametitle{Ensemble Spectrum}

\begin{alertblock}{\deftitle{Ensemble Spectrum}}
Let $\Ens \sim \D$ be an ensemble of matrices $P_i \in \F^{n \times n}$. To take the spectrum of $\Ens$, simply take the union of the spectra of each of its matrices.
In other words, if $\Ens = \{P_i \sim \mathcal{D}\}_{i = 1}^K$, then we denote the spectrum of the ensemble
$$\sigma(\Ens) = \bigcup_{i=1}^K \sigma(P_i)$$
\end{alertblock}

\end{frame}

% %=========================================================================================
% \begin{frame}
% \frametitle{Empty Slide}
% \end{frame}
% %=========================================================================================
% \begin{frame}
% \frametitle{Empty Slide}
% \end{frame}
% %=========================================================================================
% \begin{frame}
% \frametitle{Empty Slide}
% \end{frame}


%=========================================================================================
%=========================================================================================


%=========================================================================================
%   Spectrum Analysis
%=========================================================================================
\section{Spectrum Analysis}
%=========================================================================================



%=========================================================================================
%=========================================================================================






%=========================================================================================
%   Symmetric & Hermitian Matrices
%=========================================================================================
\section{Symmetric $\&$ Hermitian Matrices}
%=========================================================================================

\begin{frame} \frametitle{Wigner's Semicircle Distribution}

\begin{alertblock}{\deftitle{Wigner's Semicircle Distribution}}
If a random variable $X$ is semicircle distributed with radius $R \in \R^+$, then we say $X \sim \text{SC}(R)$. $X$ has the following probability density function:
$$\Prb(X = x) = \frac{2}{\pi R^2} \sqrt{R^2 - x^2} \for x \in [-R, R]$$
\end{alertblock}

\end{frame}
%=========================================================================================



%=========================================================================================
%=========================================================================================






%=========================================================================================
%   A Survey Of Spectra
%=========================================================================================
\section{A Survey Of Spectra}
%=========================================================================================


%=========================================================================================
%=========================================================================================
