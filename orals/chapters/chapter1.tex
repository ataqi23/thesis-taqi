
%*****************************************************************************************
%   Chapter 1
%*****************************************************************************************

%=========================================================================================
%   D-Distributions
%=========================================================================================
\section{$\D$-distributions}
%=========================================================================================
\begin{frame}
\frametitle{$\D$-distributions}

To simulate random matrices, we need to initialize their entries. But, how do we sample the random entries?

\begin{enumerate}
  \item Random Variables: Sample from a probability distribution $(X \sim \D)$
  \item Random Matrices: Define a new framework containing information about entry distributions: $\D$-distributions
    \begin{enumerate}
      \item Explicit distributions: sample (some or all) of the entries as independent random variables
      \item Implicit distributions: use an algorithm to randomly generate the matrix.
    \end{enumerate}
\end{enumerate}

\end{frame}
%=========================================================================================
\begin{frame} \frametitle{Homogenous Explicit $\D$-distributions}

  \begin{alertblock}{\deftitle{Homogenous Explicit $\D$-distributions}}
    Suppose $P \sim \D$ where $\D$ is a homogenous and explicit distribution. Additionally, let $\D^*$ denote the corresponding random variable analogue of $\D$.
    Then, every single entry of $P$ is an i.i.d random variable with the corresponding distribution. That is,
    $$ P \sim \D \iff \forall i,j \mid p_{ij} \sim \D^* $$
  \end{alertblock}

\end{frame}
%=========================================================================================
\begin{frame} \frametitle{Homogenous Explicit $\D$-distributions}

  \begin{examples}
  Suppose $P \sim \Normal(0,1)$ and that $P$ is a $2 \times 2$ matrix.
  Then, $p_{11}, p_{12}, p_{21}, p_{22}$ are independent, identically distributed random variables with the standard normal distribution.
  \end{examples}

\end{frame}
%=========================================================================================
\begin{frame} \frametitle{Non-Homogenous Explicit $\D$-distributions}

Now, as opposed to having independent and identically distributed entries for the whole matrix, the random matrix distribution may be slightly more complex to define. \newline

Fundamentally, we could characterize any matrix by identifying the distribution of every entry. But, we don't need to do so.

\end{frame}
%=========================================================================================
\begin{frame} \frametitle{Non-Homogenous Explicit $\D$-distributions}

  \begin{alertblock}{\deftitle{Diagonal Bands}}
    Suppose $P = (p_{ij})$ is an $N \times N$ matrix. Then, $P$ may be partitioned into $2n - 1$ rows called diagonal bands. Each band is denoted $[\rho]_P$ where $[\rho]_P = \{p_{ij} \mid \rho = i - j\}$. We have
    $\rho \in \{ -(N-1), \dots, -1, 0, 1, \dots, N-1 \}$.
  \end{alertblock}

\end{frame}
%=========================================================================================
\begin{frame} \frametitle{The Hermite $\beta$-Matrix}

  \begin{alertblock}{\deftitle{$\b$-matrix}}
    Suppose $P \sim \H(\b)$ is an $N \times N$ matrix. Then, the main diagonal $[0]_P \sim \Normal(0,2)$.
    Additionally, both the main off-digaonals are equal and they are given by $[1]_{P} = [-1]_{P} = \vec{X} = (X_k)_{k=1}^{N-1}$ where $X_k \sim \chi(\text{df} = \beta k)$.
    As such, we obtain a Hermite-$\b$ distributed matrix. Note that this is a symmetric tridiagonal matrix.
  \end{alertblock}

\end{frame}
%=========================================================================================
\begin{frame} \frametitle{The Hermite $\beta$-Matrix}

%FFFFFFFFFFFFFFFFFFFFFFFFFFFFFFFFFFFFFFFFFFFFFFFFFFFFFFFFFFFFFFFFFFFFFFFFFFF
\plotwrapperNC{h}{0.25}{graphics/dumitriu}%{Eigenvalues of a Symmetric Matrix displaying the Semicircle Distribution}
%FFFFFFFFFFFFFFFFFFFFFFFFFFFFFFFFFFFFFFFFFFFFFFFFFFFFFFFFFFFFFFFFFFFFFFFFFFF

\end{frame}
%=========================================================================================
\begin{frame} \frametitle{Implicit Distributions}

\begin{enumerate}
  \item Stochastic Matrices: we will generate transition matrices representing a walk on a fully connected graph with \textbf{randomized weights}.
  \item Erdos-Renyi $p$-Matrices: stochastic matrices with randomized weights, but parameterized probability $p$ of observing severed edges (zero weight).
\end{enumerate}

\end{frame}
%=========================================================================================


%=========================================================================================
%   D-Distributions
%=========================================================================================
\section{Random Matrices}
%=========================================================================================
\begin{frame} \frametitle{Random Matrices}

\begin{alertblock}{Random Matrices}
Let $P \sim \D$ be an $N \times N$ matrix over $\F$. Then, the entries of $P$ are elements in $\F$ completely determined by the $\D$-distribution, regardless of what type it is.
Also, if $\D$ is an explicit distribution, $\D^\dagger$ represents the symmetric/hermitian version of $\D$.
\end{alertblock}
\end{frame}

%=========================================================================================
\begin{frame} \frametitle{Random Matrix Ensembles}

\begin{alertblock}{\deftitle{Random Matrix Ensembles}}
A $\D$-distributed ensemble $\Ens$ of $N \times N$ random matrices over $\F$ of size $K$ is defined as a set of $K$ iterations of that class of random matrix, and it is denoted:
$$ \Ens = \bigcup_{i = 1}^K P_i \where P_i \sim \mathcal{D} \and P_i \in \F^{N \times N} $$
\end{alertblock}

\end{frame}
%=========================================================================================
\begin{frame} \frametitle{Summary of $\D$-distributions}
  \begin{center}
    \Ddisttable
  \end{center}
\end{frame}

%=========================================================================================
\begin{frame} \frametitle{Checkpoint}

Now, we have defined random matrices. We can simulate various random matrix ensembles, and are ready to analyze their spectral statistics!
\end{frame}

%=========================================================================================
