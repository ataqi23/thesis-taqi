
\chapter{Code Appendix}

%=========================================================================================
%=========================================================================================
\blocktitle{The RMAT Package}
%=========================================================================================

\noindent
The RMAT package is composed of two primary modules, the random matrix module (matrices.R) and the spectral statistics module (spectrum.R) and (dispersion.R).
They can be further subdivided into smaller modules as follows.

\blocktitle{Note} The source code was functional as of R $4.0.5$.

\begin{enumerate}
  \item \textbf{Random Matrix Module}
    \begin{enumerate}
      \item Explicitly Distributed Matrices
      \item Implicitly Distributed Matrices
      \item Ensemble Extensions
    \end{enumerate}

  \item \textbf{Spectral Statistics Module}
    \begin{enumerate}
      \item Spectrum
      \item Dispersions
      \item Parallel Extensions
      %\item Visualizations
    \end{enumerate}
\end{enumerate}

%=========================================================================================
%=========================================================================================
\newpage
%=========================================================================================
\section{Matrix Module}
%=========================================================================================

Random matrices can either be explicitly or implicitly distributed. If they are explicitly distributed, their entries have a specific distribution. Otherwise, the entries have an implicit distribution imposed by the generative algorithm the matrix uses.

\subsection{Explicitly Distributed Matrices}

For (homogeneous) explicitly distributed matrices, we can use a ``function factory'' method to be concise. The actual implementation is more verbose for the purposes of argument documentation, but the following code is minimal and fully functional. %Additionally, there are the beta matrices, which use the matrix model provided by the algorithm in Dimitriu's paper.

%******************************************************
\minititle{Homogeneously Explicit Distributions}
%******************************************************

%CCCCCCCCCCCCCCCCCCCCCCCCCCCCCCCCCCCCCCCCCCCCCCCCCCCCCCCCCCCCCCCCCCCCCCCCCCCCCCCCCCCCCCCCC
\codechunk{h}{0.9}{appendix/code_chunks/M/ME-RM_dist}
%CCCCCCCCCCCCCCCCCCCCCCCCCCCCCCCCCCCCCCCCCCCCCCCCCCCCCCCCCCCCCCCCCCCCCCCCCCCCCCCCCCCCCCCCC
\trimm
With our function factories set up, we can quickly generate all the random matrix functions for all the distributions our hearts could desire.

%CCCCCCCCCCCCCCCCCCCCCCCCCCCCCCCCCCCCCCCCCCCCCCCCCCCCCCCCCCCCCCCCCCCCCCCCCCCCCCCCCCCCCCCCC
\codechunk{h}{0.9}{appendix/code_chunks/M/ME-example}
%CCCCCCCCCCCCCCCCCCCCCCCCCCCCCCCCCCCCCCCCCCCCCCCCCCCCCCCCCCCCCCCCCCCCCCCCCCCCCCCCCCCCCCCCC
\trimm
\newpage

%******************************************************
\minititle{Beta Matrices}
%******************************************************
For the $\beta$-ensemble matrices, we simply use the algorithm provided in Dimitriu's paper. Doing so, we get the function:
%CCCCCCCCCCCCCCCCCCCCCCCCCCCCCCCCCCCCCCCCCCCCCCCCCCCCCCCCCCCCCCCCCCCCCCCCCCCCCCCCCCCCCCCCC
\codechunk{h}{0.9}{appendix/code_chunks/M/ME-RM_beta}
%CCCCCCCCCCCCCCCCCCCCCCCCCCCCCCCCCCCCCCCCCCCCCCCCCCCCCCCCCCCCCCCCCCCCCCCCCCCCCCCCCCCCCCCCC
\trimmm
%******************************************************
\minititle{Helper Functions}
%******************************************************
%CCCCCCCCCCCCCCCCCCCCCCCCCCCCCCCCCCCCCCCCCCCCCCCCCCCCCCCCCCCCCCCCCCCCCCCCCCCCCCCCCCCCCCCCC
\codechunk{h}{0.9}{appendix/code_chunks/M/MI-makeHerm}
%CCCCCCCCCCCCCCCCCCCCCCCCCCCCCCCCCCCCCCCCCCCCCCCCCCCCCCCCCCCCCCCCCCCCCCCCCCCCCCCCCCCCCCCCC

%=========================================================================================
\newpage
\subsection{Implicitly Distributed Matrices}

% In the case of implicitly distributed matrices, we have various types of stochastic matrices.

%******************************************************
\blocktitle{Stochastic Matrices} For stochastic matrices, we require slightly more setup. First, we setup the row functions to sample probability vectors:
\trim
%CCCCCCCCCCCCCCCCCCCCCCCCCCCCCCCCCCCCCCCCCCCCCCCCCCCCCCCCCCCCCCCCCCCCCCCCCCCCCCCCCCCCCCCCC
\codechunk{h}{0.9}{appendix/code_chunks/M/MI-stoch_row}
%CCCCCCCCCCCCCCCCCCCCCCCCCCCCCCCCCCCCCCCCCCCCCCCCCCCCCCCCCCCCCCCCCCCCCCCCCCCCCCCCCCCCCCCCC
\trimm
For randomly introduced sparsity, we define the following row function.
\trim
%CCCCCCCCCCCCCCCCCCCCCCCCCCCCCCCCCCCCCCCCCCCCCCCCCCCCCCCCCCCCCCCCCCCCCCCCCCCCCCCCCCCCCCCCC
\codechunk{h}{0.9}{appendix/code_chunks/M/MI-stoch_row_zeros}
%CCCCCCCCCCCCCCCCCCCCCCCCCCCCCCCCCCCCCCCCCCCCCCCCCCCCCCCCCCCCCCCCCCCCCCCCCCCCCCCCCCCCCCCCC
\trimm
Once this is done, we can use this function iteratively. With some magic, we can incorporate an option to make the matrix symmetric, and we get the following function.
%CCCCCCCCCCCCCCCCCCCCCCCCCCCCCCCCCCCCCCCCCCCCCCCCCCCCCCCCCCCCCCCCCCCCCCCCCCCCCCCCCCCCCCCCC
\codechunk{h}{0.9}{appendix/code_chunks/M/MI-RM_stoch}
%CCCCCCCCCCCCCCCCCCCCCCCCCCCCCCCCCCCCCCCCCCCCCCCCCCCCCCCCCCCCCCCCCCCCCCCCCCCCCCCCCCCCCCCCC
\trimm
%******************************************************
\newpage
\blocktitle{Erdos-Renyi Stochastic Matrices} For the Erdos-Renyi walks, we do something similar by defining a parameterized row function.
\trim
%CCCCCCCCCCCCCCCCCCCCCCCCCCCCCCCCCCCCCCCCCCCCCCCCCCCCCCCCCCCCCCCCCCCCCCCCCCCCCCCCCCCCCCCCC
\codechunk{h}{0.9}{appendix/code_chunks/M/MI-stoch_row_erdos}
%CCCCCCCCCCCCCCCCCCCCCCCCCCCCCCCCCCCCCCCCCCCCCCCCCCCCCCCCCCCCCCCCCCCCCCCCCCCCCCCCCCCCCCCCC
\trimm
And we again use the row function iteratively to get the following function.
\trim
%CCCCCCCCCCCCCCCCCCCCCCCCCCCCCCCCCCCCCCCCCCCCCCCCCCCCCCCCCCCCCCCCCCCCCCCCCCCCCCCCCCCCCCCCC
\codechunk{h}{0.9}{appendix/code_chunks/M/MI-RM_erdos}
%CCCCCCCCCCCCCCCCCCCCCCCCCCCCCCCCCCCCCCCCCCCCCCCCCCCCCCCCCCCCCCCCCCCCCCCCCCCCCCCCCCCCCCCCC
\trimm
And as such, we have minimal, functional implementations of functions that sample random matrices! %In total, we only needed two helper functions. The `.offdiagonalEntries` function was used to normalize the probabilities in `RM-stoch` and `RM-erdos`.
%=========================================================================================
\newpage
\subsection{Ensemble Extensions}

Lastly, we have the ensemble extensions. These functions are quite simple to implement user a ``function factory''. Again, the actual implementations are more verbose due to the argument descriptions, but otherwise, are exactly the same.
\trim
%CCCCCCCCCCCCCCCCCCCCCCCCCCCCCCCCCCCCCCCCCCCCCCCCCCCCCCCCCCCCCCCCCCCCCCCCCCCCCCCCCCCCCCCCC
\codechunk{h}{0.9}{appendix/code_chunks/M/MEE-RME_extender}
%CCCCCCCCCCCCCCCCCCCCCCCCCCCCCCCCCCCCCCCCCCCCCCCCCCCCCCCCCCCCCCCCCCCCCCCCCCCCCCCCCCCCCCCCC
\trimm
Now, we extend the functions as follows, and we are done with the matrix module!
\trim
%CCCCCCCCCCCCCCCCCCCCCCCCCCCCCCCCCCCCCCCCCCCCCCCCCCCCCCCCCCCCCCCCCCCCCCCCCCCCCCCCCCCCCCCCC
\codechunk{h}{0.9}{appendix/code_chunks/M/MEE-example}
%CCCCCCCCCCCCCCCCCCCCCCCCCCCCCCCCCCCCCCCCCCCCCCCCCCCCCCCCCCCCCCCCCCCCCCCCCCCCCCCCCCCCCCCCC

%=========================================================================================
\newpage
%=========================================================================================
\section{Spectral Statistics Module}
%=========================================================================================

\subsection{Spectrum}
\trim
%CCCCCCCCCCCCCCCCCCCCCCCCCCCCCCCCCCCCCCCCCCCCCCCCCCCCCCCCCCCCCCCCCCCCCCCCCCCCCCCCCCCCCCCCC
\codechunk{h}{0.80}{appendix/code_chunks/S/S_def}
%CCCCCCCCCCCCCCCCCCCCCCCCCCCCCCCCCCCCCCCCCCCCCCCCCCCCCCCCCCCCCCCCCCCCCCCCCCCCCCCCCCCCCCCCC
\trimmm \trimm
%CCCCCCCCCCCCCCCCCCCCCCCCCCCCCCCCCCCCCCCCCCCCCCCCCCCCCCCCCCCCCCCCCCCCCCCCCCCCCCCCCCCCCCCCC
\codechunk{h}{0.80}{appendix/code_chunks/S/S_def2}
%CCCCCCCCCCCCCCCCCCCCCCCCCCCCCCCCCCCCCCCCCCCCCCCCCCCCCCCCCCCCCCCCCCCCCCCCCCCCCCCCCCCCCCCCC
\trimmm \trimm
%CCCCCCCCCCCCCCCCCCCCCCCCCCCCCCCCCCCCCCCCCCCCCCCCCCCCCCCCCCCCCCCCCCCCCCCCCCCCCCCCCCCCCCCCC
\codechunk{h}{0.80}{appendix/code_chunks/S/S_def3}
%CCCCCCCCCCCCCCCCCCCCCCCCCCCCCCCCCCCCCCCCCCCCCCCCCCCCCCCCCCCCCCCCCCCCCCCCCCCCCCCCCCCCCCCCC
\trimmm \trim

\newpage
\trimmm
%******************************************************
\minititle{Helper Functions}
%******************************************************
%CCCCCCCCCCCCCCCCCCCCCCCCCCCCCCCCCCCCCCCCCCCCCCCCCCCCCCCCCCCCCCCCCCCCCCCCCCCCCCCCCCCCCCCCC
\codechunk{h}{0.9}{appendix/code_chunks/S/S_helper}
%CCCCCCCCCCCCCCCCCCCCCCCCCCCCCCCCCCCCCCCCCCCCCCCCCCCCCCCCCCCCCCCCCCCCCCCCCCCCCCCCCCCCCCCCC
\trimm
%=========================================================================================
\newpage
\subsection{Dispersions}
\trim
%CCCCCCCCCCCCCCCCCCCCCCCCCCCCCCCCCCCCCCCCCCCCCCCCCCCCCCCCCCCCCCCCCCCCCCCCCCCCCCCCCCCCCCCCC
\codechunk{h}{0.80}{appendix/code_chunks/D/D_def1}
%CCCCCCCCCCCCCCCCCCCCCCCCCCCCCCCCCCCCCCCCCCCCCCCCCCCCCCCCCCCCCCCCCCCCCCCCCCCCCCCCCCCCCCCCC
\trimmm \trimm
%CCCCCCCCCCCCCCCCCCCCCCCCCCCCCCCCCCCCCCCCCCCCCCCCCCCCCCCCCCCCCCCCCCCCCCCCCCCCCCCCCCCCCCCCC
\codechunk{h}{0.80}{appendix/code_chunks/D/D_def2}
%CCCCCCCCCCCCCCCCCCCCCCCCCCCCCCCCCCCCCCCCCCCCCCCCCCCCCCCCCCCCCCCCCCCCCCCCCCCCCCCCCCCCCCCCC
\trimmm \trimm
%CCCCCCCCCCCCCCCCCCCCCCCCCCCCCCCCCCCCCCCCCCCCCCCCCCCCCCCCCCCCCCCCCCCCCCCCCCCCCCCCCCCCCCCCC
\codechunk{h}{0.80}{appendix/code_chunks/D/D_def3}
%CCCCCCCCCCCCCCCCCCCCCCCCCCCCCCCCCCCCCCCCCCCCCCCCCCCCCCCCCCCCCCCCCCCCCCCCCCCCCCCCCCCCCCCCC
\newpage
%******************************************************
\minititle{Helper Functions}
%******************************************************
%CCCCCCCCCCCCCCCCCCCCCCCCCCCCCCCCCCCCCCCCCCCCCCCCCCCCCCCCCCCCCCCCCCCCCCCCCCCCCCCCCCCCCCCCC
\codechunk{h}{0.9}{appendix/code_chunks/D/D_helper}
%CCCCCCCCCCCCCCCCCCCCCCCCCCCCCCCCCCCCCCCCCCCCCCCCCCCCCCCCCCCCCCCCCCCCCCCCCCCCCCCCCCCCCCCCC


%=========================================================================================

%\subsection{Parallel Extensions}

%=========================================================================================
%=========================================================================================
