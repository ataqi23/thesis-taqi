
  \chapter*{Conclusion}
       \addcontentsline{toc}{chapter}{Conclusion}
\chaptermark{Conclusion}
\markboth{Conclusion}{Conclusion}

\medskip

In this thesis, the concepts of random matrix distribution, ordered spectra, and eigenvalue dispersions were rigourously formalized.
While those formulations were inspired in part by the pragmatic demand of making rigourous the mathematical methods of the RMAT package, they by no means are restricted to the domain of the package. Namely, the concepts of $\D$-distributions and eigenvalue ordering are timelessly critical concepts to understand in order to think about random matrices. Furthermore, carefully choosing dispersion metrics and selecting subsets of eigenvalue pairs in a systemic fashion is a necessary hurdle one must cross before even starting to study dispersions. \newline

All in all, this toolkit of definitions and formalizations is a toolkit that will make unambiguous any sort of communication involving random matrices. While these definitions appear polished in this thesis, they were always seeds of something smaller. So, do not expect to understand everything immediately. The document does try to accomodate this by sprinkling thoughtful examples, demonstrations, and remarks, so be sure to leave no stone unturned. So for having reached the end, the reader should be proud. You are at least more ready to delve in the deep waters of of Random Matrix Theory. \newline

Other than the formalizations, in this thesis we survey several results through analyzing and visualizing simulations. In Chapter 2, we provide a heurestic demonstration of the Perron-Frobenius Theorem for Markov Chains by using computational evidence. Also, we survey symmetric/hermitian matrices and empirically describe their spectra. We then take in a case study of what turns out to be the GOE ($\b = 1$). In Chapter 3, after we formalize dispersions, we simulate Wigner's surmise - a statement about the distribution of normalized consecutive eigenvalues of symmetric matrices. We simulate the distributions for the three standard $\b$-ensembles before we launch into Chapter 4. \newline

In Chapter 4, we bring everything together and make a case study out of the $\b$-ensembles. After discussing the Wigner's surmise result in the end of Chapter 3, we formally define and motivate the $\b$-ensemble by its joint eigenvalue p.d.f. Then, we discuss more abstract characterizations of the ensembles such as the invariance criterion and the physical interpretation of the model. The chapter ends by surveying some spectral statistics and extending the Wigner's surmise model for any $\b \in \N$.

%=========================================================================================
\newpage
%=========================================================================================
\minititle{Open Questions}

%=========================================================================================
\medskip
%=========================================================================================
\blocktitle{Spectral Statistics} What are the spectral statistics of the following $\D$-distributions?

\begin{enumerate}
  \item The following e.h. distributions
    \begin{enumerate}
      \item Poisson
      \item Beta
      \item Gamma
    \end{enumerate}
  \item Band matrices with various $\D$-distributions.
    \begin{enumerate}
      \item General $k$-band matrices
      \item Tridiagonals ($k = 1$)
    \end{enumerate}
\end{enumerate}

%=========================================================================================
\medskip
%=========================================================================================
\blocktitle{Dispersions} For the generalized $\b$-ensembles, what are the moments of the Wigner Dispersion distributions?

%=========================================================================================
\medskip
%=========================================================================================
\blocktitle{Other} How can Wishart matrices (non-square Normal matrices) be studied?
Generally, the approach to doing so is by using singular values. However, the question remains, what are their spectral statistics?
