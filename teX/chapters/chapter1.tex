\chapter{Random Matrices}

As discussed in the introduction, this thesis will be an exploration of spectral statistics of random matrices. This means that we must first be able to understand what random matrices are. At a fundamental level, random matrices are simply matrices whose entries are randomly distributed in accordance to some distribution or method. To formalize all these notions, we will define what random matrices are and what it means for them to be $\D$-distributed.

Prior to beginning the discussion on $\D$-distributions, the reader should be familiar or at least accquainted with the notion of random variables and what they are. A summary is available in the appendix in A.x.

%\section{Introduction}
When it comes to random simulation, there must always be a rule to which our randomness must conform, regardless of complexity. For example, sampling a vector from a distribution is a rudimentary example of this. For random matrices, there will be a few methods of generating their entries that are not just sampling from theoretical distributions. As such, we motivate the $\D$-distribution.

%%%%%%%%%%%%%%%%%%%%%%%%%%%%%%%%%%%%%%%%%%%%%%%%%%%%%%%%%%%%%%%%%%%%%%%%%%%%%%%%%%%%%%%%%%
%%%%%%%%%%%%%%%%%%%%%%%%%%%%%%%%%%%%%%%%%%%%%%%%%%%%%%%%%%%%%%%%%%%%%%%%%%%%%%%%%%%%%%%%%%

\section{$\D$-Distributions}

Now, we motivate the $D$-distribution. A formalization on how to initialize a random matrix.

\begin{definition}[$\D$-distribution]
When we define a random matrix that is $\D$-distributed, we say that $P \sim \D$. In the simplest of terms, $\D$ is essentially the algorithm that generates the random matrix $P$. We define two primary methods of distribution: explicit distribution and implicit distribution. If $\D$ is an explicit distribution, then every entry of $P$ is homogenously sampled from that distribution. Otherwise, if it is implicit, we utilize an algorithm that imposes an implicit distribution on the entries.
\end{definition}

%%%%%%%%%%%%%%%%%%%%%%%%%%%%%%%%%%%%%%%%%%%%%%%%%%%%%%%%%%%%%%%%%%%%%%%%%%%%%%%%%%%%%%%%%%

\subsection{Explicit Distributions}

The simplest case is homogenous, explicitly distributed random matrices. If $\D$ is an explicit distribution, then we overload the notation $\D$ to mean a probability distribution in the classical sense (see Appendix A.x). So, if $\D$ is a probability distribution, the matrix $P \sim \D$ when $p_{ij} \sim \D$. In otherwords, we simply perform entry-wise sampling from that distribution.

Take for example the following explicit distributions which we cover in this thesis.

\begin{enumerate}
\item  If $\D = \Normal(0,1)$, then $p_{ij} \sim \Normal(0,1)$.
\item  If $\D = \Unif(0,1)$, then $p_{ij} \sim \Unif(0,1)$.
\end{enumerate}

\begin{remark}[Formalization]
Explicit distributions can be formalized in scope of the standard notation in probability theory. At the heart of the formalization is the usage of index hacking to collapse the array's indices from two dimensions to one. This way, our random matrix has a representation as a random vector, which we commonly encounter in probability theory as a (i.i.d) sequence of random variables! Generally, an $N \times N$ random matrix that is explicity and homogenously $\D$-distributed is essentially a sequence of $N^2$ i.i.d random variables sampled from $\D$.
\end{remark}

\begin{example}[Formalization]
For example, suppose $P$ is a $2 \times 2$ random matrix with $\D = \Normal(0,1)$. Then, we have four random variables to initialize. In the random matrix representation, we need to initialize $P_{11}, P_{12}, P_{21}, \and P_{22}$ by sampling them from $\D$. In the vector representation, we just say we are sampling four i.i.d random variables from $\D$. The matrix indexing is intrinsic and preservable by using index hacking with some modular math.
\end{example}

%%%%%%%%%%%%%%%%%%%%%%%%%%%%%%%%%%%%%%%%%%%%%%%%%%%%%%%%%%%%%%%%%%%%%%%%%%%%%%%%%%%%%%%%%%

\subsection{Implicit Distributions}

In the latter case, we are concerned less about the distribution of the matrix entries and moreso about its holisitic properties.

Consider for example, the following implicit distributions.

\begin{enumerate}
\item If $\D = \text{Stochastic}$, then the matrix is a row of random stochastic rows. (See Algorithm B.x)
\item If $\D$ is any distribution (implicit or explicit), then $\D^{\dagger}$ is the Symmetric/Hermitian version of $\D$. (See Algorithm B.x)
\end{enumerate}

%%%%%%%%%%%%%%%%%%%%%%%%%%%%%%%%%%%%%%%%%%%%%%%%%%%%%%%%%%%%%%%%%%%%%%%%%%%%%%%%%%%%%%%%%%

\subsection{Random Matrices}

\begin{definition}[Random Matrix]
Assuming $\D$ is an explicit distribution, a random matrix is any matrix over the field $\F$ is a matrix $M \in \F^{N \times N}$ is a matrix whose entries are i.i.d random variables. So, if a random matrix $M = (m_{ij})$ is $\mathcal{D}$-distributed, then we say $m_{ij} \sim \mathcal{D}$. In the scope of this thesis, assume every random matrix to be homogenously distributed. Otherwise, if $\D$ is an implicit distribution, then $P$ is a matrix whose entries are determined by the algorithm imposed by $\D$.
\end{definition}

Sometimes, we want our matrix to have complex entries. We notate this by specifying $\F = \Cc$.

\begin{remark}[Complex Entries]
To say that a random matrix is explicitly $\D$-distributed over $\Cc$ would mean that its entries take the form $a + bi$ where $a,b \sim \D$ are random variables. In other words, if we allow the matrix to have complex entries by setting $\F = \Cc$, then we must sample the real and imaginary component as $\D$-distributed i.i.d. random variables.
\end{remark}

Below, we can see code on how to generate a standard normal random matrix using the $\textbf{RMAT}$ package.

\begin{code}[Standard Normal Matrix]
Let $\mathcal{D} = \Normal(0,1)$. We can generate $P \sim \D$, a $4 \times 4$ standard normal matrix, as such:
\end{code}

\begin{lstlisting}[language=R]
library(RMAT)
P <- RM_norm(N = 4, mean = 0, sd = 1)
# Outputs the following
P
           [,1]       [,2]       [,3]        [,4]
[1,]  0.1058257 -1.0835598 -0.7031727  1.01608625
[2,] -0.2170453  1.8206070 -0.4539230  0.06828296
[3,]  1.3002145  0.1254992 -0.5214005 -0.61516174
[4,] -1.0398587  0.1975445 -0.8511950  0.86366082
\end{lstlisting}

%****************************************************************************************%
\minititle{Summary Table of $\D$-Distributions}
\begin{center}
  \Ddisttable
\end{center}
%****************************************************************************************%
\newpage

%%%%%%%%%%%%%%%%%%%%%%%%%%%%%%%%%%%%%%%%%%%%%%%%%%%%%%%%%%%%%%%%%%%%%%%%%%%%%%%%%%%%%%%%%%
%%%%%%%%%%%%%%%%%%%%%%%%%%%%%%%%%%%%%%%%%%%%%%%%%%%%%%%%%%%%%%%%%%%%%%%%%%%%%%%%%%%%%%%%%%

\section{The Crew: Ensembles}

With a random matrix well defined, we may now motivate one of the most important ideas - the random matrix ensemble. One common theme in this thesis will be that random matrices on their own provide little information. When we consider them at the ensemble level, we start to obtain more fruitful results. Without further ado, we motivate the random matrix ensemble.

\begin{definition}[Random Matrix Ensemble]
A $\D$-distributed random matrix ensemble $\Ens$ over $\F^{N \times N}$ of size $K$ is defined as a set of $\D$-distributed random matrices $\Ens = \{P_i \sim \mathcal{D} \mid P_i \in \F^{N \times N}\}_{i = 1}^K$. In simple words, it is simply a collection of $K$ iterations of a specified class of random matrix.
\end{definition}

So, for example, we could compute a simple ensemble of matrices as follows.

\begin{code}[Standard Normal Hermitian Ensemble]
Let $\mathcal{D} = \Normal(0,1)^{\dagger}$. We can generate $\Ens \sim \D$ over $\Cc$, an ensemble of $4 \times 4$ complex Hermitian standard normal matrices of size 10 as such:
\end{code}

\begin{lstlisting}[language=R]
library(RMAT)
# By default, mean = 0 and sd = 1.
ensemble <- RME_norm(N = 4, cplx = TRUE, herm = TRUE, size = 10)
# Outputs the following
ensemble
...
[[10]]
                  [,1]              [,2]              [,3]
[1,] -0.59931+1.24286i  1.29457+0.66058i  0.83539-0.16662i
[2,]  1.29457-0.66058i  0.78841+0.09818i -1.16592+1.14666i
[3,]  0.83539+0.16662i -1.16592-1.14666i -0.51256+0.17750i
\end{lstlisting}

With this in mind, we will gloss over, characterize, and briefly discuss a few special recurring ensembles in this thesis.

%%%%%%%%%%%%%%%%%%%%%%%%%%%%%%%%%%%%%%%%%%%%%%%%%%%%%%%%%%%%%%%%%%%%%%%%%%%%%%%%%%%%%%%%%%

\subsection{Hermite $\beta$-Ensembles}

The Hermite $\beta$-Ensembles will be one of the primary ensembles discussed in this thesis. This ensemble will be characterized, motivated, and defined more thoroughly in $\textbf{Chapter 4}$. However, we will give a brief introduction to the ensemble.

At a practical level, all one would need to know is that the matrices are generated in accordance to an algorithm found in the appendix (Algorithm B.x) and in Chapter 4.

%%%%%%%%%%%%%%%%%%%%%%%%%%%%%%%%%%%%%%%%%%%%%%%%%%%%%%%%%%%%%%%%%%%%%%%%%%%%%%%%%%%%%%%%%%

\subsection{Erdos-Renyi $p$-Ensembles}

The Hermite $\beta$-ensembles are a normal-like class of random matrices. Now, we will veer away from the normal distribution as a whole and switch to a different class of matrices: stochastic matrices. Stochastic matrices, in short, are matrices that represent Markov Chains (see A.x). We can also think of them as the matrix representation of a specific setup of a walk on a random graph.

A particular class of random graphs that we will consider are the Erdos-Renyi random graphs. Essentially, these are graphs whose vertices are connected with a uniform probability $p$. We can interpret this as saying an Erdos-Renyi graph is a simple random walk on a graph with parameterized sparsity (given by $p$). Without further ado, we motivate the Erdos-Renyi graph:

\begin{definition}[Erdos-Renyi Graph]
An Erdos-Renyi graph is a graph $G = (V,E)$ with a set of vertices $V = \{\oneto[N]\}$ and edges $E = \mathds{1}_{i,j \in V} \sim \Bern(p_{ij})$. It is homogenous if $p_{ij} = p$ is fixed for all $i, j$.
\end{definition}

Essentially, an Erdos-Renyi graph is a graph whose 'connectedness' is parameterized by a probability $p$ (assuming it's homogenous, which this document will unless otherwise noted). As $p \to 0$, we say that graph becomes more sparse; analogously, as $p \to 1$ the graph becomes more connected.\newline
\indent Recall from probability theory that a sum of i.i.d Bernoulli random variables is a Binomial variable. As such, we may alternatively say that the degree of each vertex $v$ is distributed as $deg(v) \sim Bin(N,p)$ where $N$ is the number of vertices. This makes simulating the graphs much easier.

\begin{code}[Erdos-Renyi p = 0.5 Ensemble]
Let $\mathcal{D} = \text{ER}(p = 0.5)$. We can generate $\Ens \sim \D$, an ensemble of $4 \times 4$ Erdos-Renyi matrices ($p = 0.5$) of size 10 as such:
\end{code}

\begin{lstlisting}[language=R]
library(RMAT)
ensemble <- RME_erdos(N = 4, p = 0.5, size = 10)
# Outputs the following
ensemble
...
[[10]]
          [,1]      [,2]      [,3]     [,4]
[1,] 0.0000000 0.1729581 0.8270419 0.000000
[2,] 0.0000000 0.0000000 1.0000000 0.000000
[3,] 0.2557890 0.3766740 0.0000000 0.367537
[4,] 0.2151029 0.3929580 0.3919391 0.000000
\end{lstlisting}

%%%%%%%%%%%%%%%%%%%%%%%%%%%%%%%%%%%%%%%%%%%%%%%%%%%%%%%%%%%%%%%%%%%%%%%%%%%%%%%%%%%%%%%%%%

%\section{Analytical Results}
