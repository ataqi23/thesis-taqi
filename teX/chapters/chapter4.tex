
\chapter{$\beta$-Ensembles}

%=========================================================================================
\epigraph{With thermodynamics, one can calculate almost everything crudely; with kinetic theory, one can calculate fewer things, but more accurately;
and with statistical mechanics, one can calculate almost nothing exactly.}{\textit{Eugene Wigner}}
%=========================================================================================

\section{Introduction}

In this chapter, we will talk about the Hermite $\beta$-ensembles, also commonly known as the \textit{Gaussian ensembles}, more in depth.
The $\b$-ensembles have wide applications in statistical physics, eningeering, and many other places.
Canonically, there are three ``standard'' $\b$-ensembles, corrosponding to the values of $\b = 1, 2, \and 4$.
It was briefly mentioned in a previous section, but what makes these ensembles special is how they are characterized.
Fundamentally, the ensembles are defined by the joint density of their eigenvalues, dependent on the parameter $\beta$ of course.
While $\b$-ensemble model parameter $\b$ has a support of $\N$, the three standard values give us very special properties that we discuss in this chapter.
Extending the model beyond those standard values (which we do using the $\b$-matrix model) as we do in the previous sections in turn mean that the following special characterizations
that will be discussed do not apply anymore.

%=========================================================================================
\newpage
\subsection{Hermite $\beta$-Ensemble}

To begin, we will write down the joint eigenvalue probability density function of a Hermite $\b$-matrix.

\begin{definition}[$\beta$-ensemble]
A (Hermite) $\beta$-ensemble is an ensemble of random matrices parameterized by $\beta$, which determines the joint eigenvalue p.d.f that characterizes it.
So, given an observed set of eigenvalues $\L = (\seqo[N]{\lambda})$. Then, the joint p.d.f. of $\L$ is as follows:
\begin{align*}
\jepdf = C_\beta \prod_{i < j} |\lambda_i - \lambda_j|^\beta e^{-\frac{1}{2} \sum_{i=1}^N \lambda_i^2}
%f_\beta(\Lambda) = c_\beta \prod_{i < j} |\lambda_i - \lambda_j|^\beta \exp({-\frac{1}{2} \sum_{i=1}^N \lambda_i^2})
\end{align*}
where the normalization constant $C_\beta$ is given by:
\begin{align*}
C_{\beta} = (2\pi)^{-n/2} \prod_{j = 1}^n \frac{\Gamma(1 + \frac{\b}{2})}{\Gamma(1 + \frac{\b}{2}j)}
\end{align*}
\end{definition}

%=========================================================================================
\minititle{Breaking Down the Behemoth}

Now, if the reader's reaction to the statement of the $\b$-ensemble was shock or trepidation. Do not worry.
Everything will become clear soon as we start to understand the components of the density function and
focus on characterizing the components rather than try to parse through the dense notation.
Speaking of which, let us start to rewrite the density function with friendlier, more readable notation.
To begin, let us refer to the two primary components as the ``blue term'' and the ``red term'' as shown below.
\begin{center}
$\jepdf = \betaconst \; $\bbx{$\blueterm$} \; \rbx{$\redterm$}
\end{center}

\blocktitle{Normalization Constant} First things first, let us consider the normalization constant $\betaconst$.
This constant is simply a constant needed by $f_\b$ to normalize the p.d.f such that it sums to $1$.
The term is expressed as a product of gamma terms involving $\beta$. Additionally, this term is related to an integral called the \textit{Selberg integral}. (Ref)
While the constant is interesting in terms of its relationship with that integral, it is not under our perview. So, the important takeaway is that it just normalizes our p.d.f,
because every p.d.f needs to sum to $1$ in the end of the day.

\bigskip
%\newpage

\blocktitle{The Blue Term} The first big component, which we will call ``the blue term'' is the term $\blueterme$. Let us break this term down further.
  \begin{enumerate}
    \item The first thing to note is that this term is a product of terms in the form $|\l_j - \l_i|^\b$.
    \item Secondly, note that this product term runs over the indices $1 \leq i < j \leq n$.
  \end{enumerate}
    Here's the great part: the reader has already seen and is already accquianted with those two items! Realize,
  \begin{enumerate}
    \item The term $|\l_j - \l_i|^\b$ is simply the dispersion of $\piij$ with respect to the $\b$-norm dispersion metric $\d_\b$.
      So, we can rewrite $\d_\b(\piij) = |\l_j - \l_i|^\b$.
    \item The indices $\{i < j \mid i,j \in \N_N\}$ may look familiar. This is the lower pairing scheme!
  \end{enumerate}
  As such, when we consider two facts, the interpretation hopefully becomes a little simpler. We can say that the blue term is a product of the dispersion of all eigenvalue pairs in the lower pairing scheme w.r.t the $\b$-norm dispersion metric.
Now, we know that for all $\b \in \N$ that the $\b$-norm is a positive definite metric. As such, this means that the term $\d_\b(\pi)$ is a positive term that grows when the dispersion between the eigenvalues in the pair $\pi$ grow larger.
So, the important takeaway here is that the blue term tells us that the eigenvalues in the $\b$-ensemble tend to ``repel'' each other.
The farther all the eigenvalues in $\L$ are from each other, the more likely we are to observe such $\L$.

Additionally, this also means there is \textbf{zero probability} of observing any pair of identical eigenvalues.
This is because $\d_\b(z,z') = 0$ if $z = z'$, meaning that if we observed two identical eigenvalues, then the probability density would be zero.
This effectively places a bound on the how close our eigenvalues can be to each other.

\bigskip

\blocktitle{The Red Term} The second big component, which we will call ``the red term'' is the term $\redterm$. This term is slightly simpler to understand.
Fundamentally, this term is simply an exponential term with an input that is a \textbf{negative value}.

To see this, first label the term $S = \sum_{i = 1}^N \l_i^2$. This simplifies the red term to become $\exp(-\frac{1}{2} S)$.
We know the term $S$ to be positive since the eigenvalues are \textbf{real} and that the sum of squares of several real values is strictly positive.
As such, the term, multiplied by a negative number ($-\frac{1}{2}$) is guranteed to be negative. So, to summarize, we know that $-\frac{1}{2} S$ is a negative value.
Then, from what we know about the exponential function, we know that:
$$\lim_{x \to -\infty} \exp(x) = 0$$
So, this means that as $S \to \infty$, $\exp(-\frac{1}{2}S) \to 0$. Since $S$ is simply the sum of magnitudes of the eigenvalues, this implies that the larger our eigenvalues
are, the smaller is our red term. In turn, this implies that the joint p.d.f. of eigenvalues is smaller overall.
To summarize, this means that our red term tells us that (relatively speaking) the larger the eigenvalues in $\L$, the less likely we are to observe those eigenvalues in the $\b$-ensemble. This effectively places a bound on the sizes of our eigenvalues.

\bigskip

\blocktitle{Takeaways} Altogether, here is what we can say about the $\beta$-ensemble joint eigenvalue p.d.f just from observing the terms.
  \begin{enumerate}
    %\item When $\l_i \in \L$ is large, then $\P(\L)$ is small.
    %\item When $\d(\l_i, \l_j) \for \l_i,\l_j \in \L$ is small, then $\P(\L)$ is small.
    \item When $\l_i$ is large, then $\P(\L)$ is small.
    \item When $\d(\l_i, \l_j)$ is small, then $\P(\L)$ is small.
  \end{enumerate}

%=========================================================================================
%=========================================================================================
\newpage
\subsection{The Invariance Criterion}

\minititle{The Three Musketeers (Associative Algebras)}

We know from Abstract Algebra (Ref) that there are only three associative algebras (i.e. algebras with a robust multiplication structure).
They are the following:
% \begin{enumerate}
%   \item The real numbers: $$\R = \{ x \in \R \}$$
%   \item The complex numbers: $$\C = \{ z = x + yi \mid x,y \in \R \}$$
%   \item The quaternionic numbers: $$\Hh = \{ \alpha = a + bi + cj + dk \mid a,b,c,d \in \R \}$$
% \end{enumerate}
\begin{enumerate}
  \item The real numbers $\R = \{ x \in \R \}$
  \item The complex numbers $\C = \{ z = x + yi \mid x,y \in \R \}$
  \item The quaternionic numbers $\Hh = \{ \alpha = a + bi + cj + dk \mid a,b,c,d \in \R \}$
\end{enumerate}

\minititle{Diagonalization \& Conjugation Invariance}

One of the most elegant results in Random Matrix Theory is the equivalence between the $\b$-ensembles
and their normal matrix counterparts over the three associative algebras. Without futher ado, consider the following.

\medskip

\blocktitle{Real} Suppose $M \sim \Normal(\mu,\sigma^2)^\dagger$ over $\R$ is a real symmetric matrix.
Then, its eigenvalues have a joint p.d.f. of $f_{\b = 1}(\L)$, meaning it repersents the $\beta = 1$ ensemble.

Additionally, we know that symmetric matrices are diagonalizable under conjugation with a orthogonal matrix.
In other words, we could express $M$ as $P^{-1} D P$ where $D$ is a diagonal matrix and $P \in O(n)$ is an orthogonal matrix.
For this reason, we call the $\b = 1$ ensemble the \textit{Gaussian Orthogonal Ensemble}, abbreviated as GOE.

\bigskip

\blocktitle{Complex} Suppose $M \sim \Normal(\mu,\sigma^2)^\dagger$ over $\C$ is a complex hermitian matrix.
Then, its eigenvalues have a joint p.d.f. of $f_{\b = 2}(\L)$, meaning it repersents the $\beta = 2$ ensemble.

Additionally, we know that hermitian matrices are diagonalizable under conjugation with a unitary matrix.
In other words, we could express $M$ as $P^{-1} D P$ where $D$ is a diagonal matrix and $P \in U(n)$ is an unitary matrix.
For this reason, we call the $\b = 1$ ensemble the \textit{Gaussian Unitary Ensemble}, abbreviated as GUE.

\bigskip

\blocktitle{Quaternionic} Suppose $M \sim \Normal(\mu,\sigma^2)^*$ over $\Hh$ is a quaternionic self-dual matrix.
Then, its eigenvalues have a joint p.d.f. of $f_{\b = 4}(\L)$, meaning it repersents the $\beta = 4$ ensemble.

Additionally, we know that self-dual quaternionic matrices are diagonalizable under conjugation with a sympleptic matrix.
In other words, we could express $M$ as $P^{-1} D P$ where $D$ is a diagonal matrix and $P \in Sp(n)$ is a sympleptic matrix.
For this reason, we call the $\b = 1$ ensemble the \textit{Gaussian Sympleptic Ensemble}, abbreviated as GSE.

\bigskip

All that being said, we can say that with respect to each matrix group, each $\b$-ensemble matrix has a feature called
\textbf{conjugation invariance}.

\newpage
%=========================================================================================
\minititle{Dyson Index}

So, all that being said, why do we use the value $\b$ and where do the numbers come from? The answer is the Dyson index. (Ref)
The Dyson index $\beta$ corrosponds to the number of real number of components the subject matrices have. Refer to the definitions
of the associative algebras above, and you will find that for $\b = 1,2,4$, you have the representation of a matrix over a field with $\b$ real dimensions.


%=========================================================================================
%=========================================================================================
\newpage
\subsection{A Physical Interpretation}

Alongside these great algebraic properies, the $\b$-ensembles also have a intepretation as a physical model.
This is because as mentioned previously, these ensembles show up frequently in statistical physics.
So, we will cover one physical model that the $\b$-ensemble represents.

Suppose that $P \sim \H(\beta)$ is an $N \times N$ matrix. Then, the eigenvalues of $P$ have a representation as a model of charged particles.

\minititle{Charged Particle Model}

Suppose $N$ particles are in a line about the origin. Additionally, suppose there is a quadratic potential $V(x) = x^2$.
Then, the eigenvalues of $P$, $\spec(P)$ represents a stochastic system of such particles. As discussed previously in the breakdown of the density function,
these particles tend to repel each other. Specifically, their repulsion factor is $\beta$ itself! By observing the red term and blue term at extreme values of $\b$,
we can interpret the physical model as follows:
\medskip

\blocktitle{Low Replsuion, High Temperature} As $\b \to 0$, the temperature of the system $T \to \infty$.
At these values, the model starts to behave like an ideal gas.
Additionally, there is no interaction between the paricles anymore as the power of the blue term implies that the dispersion has no effect anymore. This returns a fully stochastic
model of particles trying to align themselves. In this model, the potential matters a lot, as it determines the positioning of the particles.

\bigskip

\blocktitle{High Repulsion, Low Temperature} As $\b \to \infty$, the temperature of the system $T \to 0$.
At these values, the model loses its stochastic properties and becomes deterministic.
Additionally, there is maximal interaction between the paricles as the power of the blue term starts to become more prominent. This returns a fully deterministic*
model of particles that align themselves equidistantly. In this model, the potential doesn't matter anymore.

\medskip

\begin{remark}[Deterministic System]
While the second model is effectively deterministic, there does remain the randomness in the indices of the particles which are uniform.
\end{remark}

%=========================================================================================



%=========================================================================================
\newpage
%=========================================================================================
\subsection{The Matrix Model}

Now, we have throughly charecterized, interpreted, and explained the $\b$-ensemble. How do we simulate such an ensemble given that we only know the joint p.d.f. of the eigenvalues?
This is a problem known as the inverse eigenvalue problem. (Ref) Luckily, we do not need to dwelve deep into this. To simulate matrices from the $\beta$-ensemble, we will be using the result published in ``Matrix Models for Beta Ensembles'' by Dr. Ioana Dumitriu. \cite{dumitriu}. This gives us a matrix model that generates a matrix that would be observed in a $\beta$-ensemble given any $\beta \in \N$.

\minititle{Dumitriu's Matrix Model of $\b$-Ensembles}

This result was actually described previously in \textbf{Section 1.1.1}! We first described the $\b$-matrices as a non-homogenous $\D$-distribution that we denoted $\D = \H(\b)$. This is a symmetric, tridiagonal matrix model with entries sampling from the normal and chi distributions. What we get in turn, is a matrix model whose eigenvalues \textbf{implicitly} have the joint p.d.f. of the Hermite-$\b$ ensemble described in \textbf{Section 4.1.1}. The algorithm used is directly cited from the results of Dumitriu's paper, and can be found below.

\ALGbeta

In turn, we can now easily simulate $\b$-ensembles using $\RMAT$ as such.

\begin{code}[Hermite Beta = 2 Ensemble]
Let $\mathcal{D} = \mathcal{H}(\beta = 2)$. We can generate $\Ens \sim \D$, an ensemble of $4 \times 4$ Hermite matrices ($\beta = 2$) of size 10 as such:
\end{code}

\begin{lstlisting}[language=R]
library(RMAT)
ensemble <- RME_beta(N = 4, beta = 2, size = 10)
# Outputs the following
ensemble
...
[[10]]
          [,1]      [,2]        [,3]     [,4]
[1,] 0.7246302 1.8893868  0.00000000 0.000000
[2,] 1.8893868 1.5278221  0.68840045 0.000000
[3,] 0.0000000 0.6884004 -0.03876104 1.944495
[4,] 0.0000000 0.0000000  1.94449533 1.042741
\end{lstlisting}
%=========================================================================================

\newpage
\section{Spectra}

We know that $\beta$-matrices must be symmetric. So, their eigenvalues must be real. Any imaginary component we observe is simply computational error, and we may safely ignore it.
It is good to see that the error is uniform and small.

\begin{remark}[Implicit Distribution]
Note that the beta ensemble is an ensemble characterized by some joint density function on its eigenvalues. So, this is a specific instance of an implcitly distributed matrix.
There are many things to consider about $\textbf{identifiability}$ that are subtle, but important.
Recall that in our formalization of a specturm as a formal statistic, we vectorized the matrix then tooks the determinant of the characteristic polynomial that came about it.
For this reason, we can see that while Dumitriu's model provides an explicit formula, there is an issue of identifiability.
That is, given some eigenvalues, there is no injective function to the characteristic polynomial that produced it.
Rather, there are infinitly many equivalence classes of characteristic polynomials (and such, random matrices) that surjectively produce a given multiset of eigenvalues.
\end{remark}

\begin{remark}[Floating Point Errors]
Because the model involves diving by $\sqrt{2}$, floating point errors and algorithmic systemic error will yield small, but negligible imaginary components.
This is on top of the floating errors from rounding, as discussed in \textbf{Section 2.3}.
\end{remark}

\newpage
%FFFFFFFFFFFFFFFFFFFFFFFFFFFFFFFFFFFFFFFFFFFFFFFFFFFFFFFFFFFFFFFFFFFFFFFFFFF
\FIGUREbetaREspec{h}{0.6}
%FFFFFFFFFFFFFFFFFFFFFFFFFFFFFFFFFFFFFFFFFFFFFFFFFFFFFFFFFFFFFFFFFFFFFFFFFFF

\newpage
%FFFFFFFFFFFFFFFFFFFFFFFFFFFFFFFFFFFFFFFFFFFFFFFFFFFFFFFFFFFFFFFFFFFFFFFFFFF
\FIGUREbetaNORMspec{h}{0.6}
%FFFFFFFFFFFFFFFFFFFFFFFFFFFFFFFFFFFFFFFFFFFFFFFFFFFFFFFFFFFFFFFFFFFFFFFFFFF

\newpage
%FFFFFFFFFFFFFFFFFFFFFFFFFFFFFFFFFFFFFFFFFFFFFFFFFFFFFFFFFFFFFFFFFFFFFFFFFFF
\FIGUREbetaREsummary{h}{0.6}
%FFFFFFFFFFFFFFFFFFFFFFFFFFFFFFFFFFFFFFFFFFFFFFFFFFFFFFFFFFFFFFFFFFFFFFFFFFF

\newpage
%FFFFFFFFFFFFFFFFFFFFFFFFFFFFFFFFFFFFFFFFFFFFFFFFFFFFFFFFFFFFFFFFFFFFFFFFFFF
\FIGUREbetaNORMsummary{h}{0.6}
%FFFFFFFFFFFFFFFFFFFFFFFFFFFFFFFFFFFFFFFFFFFFFFFFFFFFFFFFFFFFFFFFFFFFFFFFFFF

\newpage
%FFFFFFFFFFFFFFFFFFFFFFFFFFFFFFFFFFFFFFFFFFFFFFFFFFFFFFFFFFFFFFFFFFFFFFFFFFF
\FIGUREbetaNORMsummary{h}{0.6}
%FFFFFFFFFFFFFFFFFFFFFFFFFFFFFFFFFFFFFFFFFFFFFFFFFFFFFFFFFFFFFFFFFFFFFFFFFFF

\newpage
%FFFFFFFFFFFFFFFFFFFFFFFFFFFFFFFFFFFFFFFFFFFFFFFFFFFFFFFFFFFFFFFFFFFFFFFFFFF
\FIGUREbetaNORMsummary{h}{0.6}
%FFFFFFFFFFFFFFFFFFFFFFFFFFFFFFFFFFFFFFFFFFFFFFFFFFFFFFFFFFFFFFFFFFFFFFFFFFF

%=========================================================================================
\newpage
%=========================================================================================
%=========================================================================================
\section{Dispersions}
%=========================================================================================

Consider the following plot of the sign-sorted eigenvalue dispersions below.

%=========================================================================================
\subsection{Linear Fit}
%=========================================================================================

Consider the following plot of beta-ensemble dispersions.

\plotwrapper{h}{0.3}{../graphics/chap4/4-2_lm_beta_big}
{Beta = 4}

%=========================================================================================
\newpage
%=========================================================================================
\subsection{Wigner's Surmise}
%=========================================================================================

\minititle{Extending Beta Ensembles}

Below, we extended the simulations beyond the standard values of $\beta = 1,2,4$. We notice a continuing pattern! The densities seem to have less variance as
beta increases.

%FFFFFFFFFFFFFFFFFFFFFFFFFFFFFFFFFFFFFFFFFFFFFFFFFFFFFFFFFFFFFFFFFFFFFFFFFFF
\FIGUREwignerbetaextended{h}{0.6}
%FFFFFFFFFFFFFFFFFFFFFFFFFFFFFFFFFFFFFFFFFFFFFFFFFFFFFFFFFFFFFFFFFFFFFFFFFFF

\minititle{Plot Takeaways}

In 1 or 2 paragraphs, describe the most important observations you want your reader to take away from this plot
