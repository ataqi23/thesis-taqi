% This is the Reed College LaTeX thesis template. Most of the work 
% for the document class was done by Sam Noble (SN), as well as this
% template. Later comments etc. by Ben Salzberg (BTS). Additional
% restructuring and APA support by Jess Youngberg (JY).
% Your comments and suggestions are more than welcome; please email
% them to cus@reed.edu
%
% See http://web.reed.edu/cis/help/latex.html for help. There are a 
% great bunch of help pages there, with notes on
% getting started, bibtex, etc. Go there and read it if you're not
% already familiar with LaTeX.
%
% Any line that starts with a percent symbol is a comment. 
% They won't show up in the document, and are useful for notes 
% to yourself and explaining commands. 
% Commenting also removes a line from the document; 
% very handy for troubleshooting problems. -BTS

% As far as I know, this follows the requirements laid out in 
% the 2002-2003 Senior Handbook. Ask a librarian to check the 
% document before binding. -SN

%%
%% Preamble
%%
% \documentclass{<something>} must begin each LaTeX document
\documentclass[12pt,twoside]{reedthesis}
% Packages are extensions to the basic LaTeX functions. Whatever you
% want to typeset, there is probably a package out there for it.
% Chemistry (chemtex), screenplays, you name it.
% Check out CTAN to see: http://www.ctan.org/
%%
\usepackage{graphicx,latexsym} 
\usepackage{amssymb,amsthm,amsmath}
\usepackage{longtable,booktabs,setspace} 
\usepackage{chemarr} %% Useful for one reaction arrow, useless if you're not a chem major
\usepackage[hyphens]{url}
\usepackage{rotating}
\usepackage{natbib}
% Comment out the natbib line above and uncomment the following two lines to use the new 
% biblatex-chicago style, for Chicago A. Also make some changes at the end where the 
% bibliography is included. 
%\usepackage{biblatex-chicago}
%\bibliography{thesis}

% \usepackage{times} % other fonts are available like times, bookman, charter, palatino

\title{Eigenanalysis of Erdos-Renyi Graphs}
\author{Ali Taqi}
% The month and year that you submit your FINAL draft TO THE LIBRARY (May or December)
\date{May 2021}
\division{Mathematics and Natural Sciences}
\advisor{Jonathan M. Wells}
%If you have two advisors for some reason, you can use the following
%\altadvisor{Your Other Advisor}
%%% Remember to use the correct department!
\department{Mathematics}
% if you're writing a thesis in an interdisciplinary major,
% uncomment the line below and change the text as appropriate.
% check the Senior Handbook if unsure.
%\thedivisionof{The Established Interdisciplinary Committee for}
% if you want the approval page to say "Approved for the Committee",
% uncomment the next line
%\approvedforthe{Committee}

\setlength{\parskip}{0pt}
%%
%% End Preamble
%%
%% The fun begins:
\begin{document}

  \maketitle
  \frontmatter % this stuff will be roman-numbered
  \pagestyle{empty} % this removes page numbers from the frontmatter

% Acknowledgements (Acceptable American spelling) are optional
% So are Acknowledgments (proper English spelling)
    \chapter*{Acknowledgements}
	I want to thank a few people.

% The preface is optional
% To remove it, comment it out or delete it.
    \chapter*{Preface}
	This is an example of a thesis setup to use the reed thesis document class.
	
	

    \chapter*{List of Abbreviations}
		You can always change the way your abbreviations are formatted. Play around with it yourself, use tables, or come to CUS if you'd like to change the way it looks. You can also completely remove this chapter if you have no need for a list of abbreviations. Here is an example of what this could look like:

	\begin{table}[h]
	\centering % You could remove this to move table to the left
	\begin{tabular}{ll}
		\textbf{ABC}  	&  American Broadcasting Company \\
		\textbf{CBS}  	&  Columbia Broadcasting System\\
		\textbf{CDC}  	&  Center for Disease Control \\
		\textbf{CIA}  	&  Central Intelligence Agency\\
		\textbf{CLBR} 	&  Center for Life Beyond Reed\\
		\textbf{CUS}  	&  Computer User Services\\
		\textbf{FBI}  	&  Federal Bureau of Investigation\\
		\textbf{NBC}  	&  National Broadcasting Corporation\\
	\end{tabular}
	\end{table}
	

    \tableofcontents
% if you want a list of tables, optional
    \listoftables
% if you want a list of figures, also optional
    \listoffigures

% The abstract is not required if you're writing a creative thesis (but aren't they all?)
% If your abstract is longer than a page, there may be a formatting issue.
    \chapter*{Abstract}
	The preface pretty much says it all.
	
	\chapter*{Dedication}
	You can have a dedication here if you wish.

  \mainmatter % here the regular arabic numbering starts
  \pagestyle{fancyplain} % turns page numbering back on

%The \introduction command is provided as a convenience.
%if you want special chapter formatting, you'll probably want to avoid using it altogether

    \chapter*{Introduction}
         \addcontentsline{toc}{chapter}{Introduction}
	\chaptermark{Introduction}
	\markboth{Introduction}{Introduction}
	% The three lines above are to make sure that the headers are right, that the intro gets included in the table of contents, and that it doesn't get numbered 1 so that chapter one is 1.

% Double spacing: if you want to double space, or one and a half 
% space, uncomment one of the following lines. You can go back to 
% single spacing with the \singlespacing command.
% \onehalfspacing
% \doublespacing
	
	Welcome to the \LaTeX\ thesis template. If you've never used \TeX\ or \LaTeX\ before, you'll have an initial learning period to go through, but the results of a nicely formatted thesis are worth it for more than the aesthetic benefit: markup like \LaTeX\ is more consistent than the output of a word processor, much less prone to corruption or crashing and the resulting file is smaller than a Word file. While you may have never had problems using Word in the past, your thesis is going to be about twice as large and complex as anything you've written before, taxing Word's capabilities. If you're still on the fence about  using \LaTeX, read the Introduction to LaTeX on the CUS site as well as skim the following template and give it a few weeks. Pretty soon all the markup gibberish will become second nature.

\section{Why use it?}
	
\LaTeX\ does a great job of formatting tables and paragraphs. Its line-breaking algorithm was the subject of a PhD.\thinspace thesis. It does a fine job of automatically inserting ligatures, and to top it all off it is the only way to typeset good-looking mathematics.

\section{Who should use it?}

Anyone who needs to use math, tables, a lot of figures, complex cross-references, IPA or who just cares about the final appearance of their document should use \LaTeX. At Reed, math majors are required to use it, most physics majors will want to use it, and many other science majors may want it also.
	

\chapter{Random Matrices}

%=========================================================================================
\epigraph{Unfortunately, no one can be told what The Matrix is. You'll have to see it for yourself.}{\textit{Morpheus \\ The Matrix}}
%=========================================================================================

As discussed in the introduction, this thesis will be an exploration of spectral statistics of random matrices. This means that we must first be able to understand what random matrices are. At a fundamental level, random matrices are simply matrices whose entries are randomly distributed in accordance to some distribution or method. To formalize all these notions, we will define what random matrices are and what it means for them to be $\D$-distributed.

Prior to beginning the discussion on $\D$-distributions, the reader should be familiar or at least accquainted with the notion of random variables and what they are. A summary is available in the appendix in A.x.

%\section{Introduction}
When it comes to random simulation, there must always be a rule to which our randomness must conform, regardless of complexity. For example, sampling a vector from a distribution is a rudimentary example of this. For random matrices, there will be a few methods of generating their entries that are not just sampling from theoretical distributions. As such, we motivate the $\D$-distribution.

%=========================================================================================
%=========================================================================================

\section{$\D$-Distributions}

Now, we motivate the $D$-distribution. A formalization on how to initialize a random matrix.

\begin{definition}[$\D$-distribution]
When we define a random matrix that is $\D$-distributed, we say that $P \sim \D$. In the simplest of terms, $\D$ is essentially the algorithm that generates the random matrix $P$. We define two primary methods of distribution: explicit distribution and implicit distribution. If $\D$ is an explicit distribution, then every entry of $P$ is homogenously sampled from that distribution. Otherwise, if it is implicit, we utilize an algorithm that imposes an implicit distribution on the entries.
\end{definition}

%=========================================================================================

\subsection{Explicit Distributions}

The simplest case is homogenous, explicitly distributed random matrices. If $\D$ is an explicit distribution, then we overload the notation $\D$ to mean a probability distribution in the classical sense (see Appendix A.x). So, if $\D$ is a probability distribution, the matrix $P \sim \D$ when $p_{ij} \sim \D$. In otherwords, we simply perform entry-wise sampling from that distribution.

Take for example the following explicit distributions which we cover in this thesis.

\begin{enumerate}
\item  If $\D = \Normal(0,1)$, then $p_{ij} \sim \Normal(0,1)$.
\item  If $\D = \Unif(0,1)$, then $p_{ij} \sim \Unif(0,1)$.
\end{enumerate}

\begin{formalization}
Explicit distributions can be formalized in scope of the standard notation in probability theory. At the heart of the formalization is the usage of index hacking to collapse the array's indices from two dimensions to one. This way, our random matrix has a representation as a random vector, which we commonly encounter in probability theory as a (i.i.d) sequence of random variables! Generally, an $N \times N$ random matrix that is explicity and homogenously $\D$-distributed is essentially a sequence of $N^2$ i.i.d random variables sampled from $\D$.
\end{formalization}

\begin{example}[Formalization]
For example, suppose $P$ is a $2 \times 2$ random matrix with $\D = \Normal(0,1)$. Then, we have four random variables to initialize. In the random matrix representation, we need to initialize $P_{11}, P_{12}, P_{21}, \and P_{22}$ by sampling them from $\D$. In the vector representation, we just say we are sampling four i.i.d random variables from $\D$. The matrix indexing is intrinsic and preservable by using index hacking with some modular math.
\end{example}

%=========================================================================================

\subsection{Implicit Distributions}

In the latter case, we are concerned less about the distribution of the matrix entries and moreso about its holisitic properties.

Consider for example, the following implicit distributions.

\begin{enumerate}
\item If $\D = \text{Stochastic}$, then the matrix is a row of random stochastic rows. (See Algorithm B.x)
\item If $\D$ is any distribution (implicit or explicit), then $\D^{\dagger}$ is the Symmetric/Hermitian version of $\D$. (See Algorithm B.x)
\end{enumerate}

%=========================================================================================

\subsection{Random Matrices}

With $\D$-distributions defined, we may finally define a random matrix.

\begin{definition}[Random Matrix]
Assuming $\D$ is an explicit distribution, a random matrix is any matrix over the field $\F$ is a matrix $M \in \F^{N \times N}$ is a matrix whose entries are i.i.d random variables. So, if a random matrix $M = (m_{ij})$ is $\mathcal{D}$-distributed, then we say $m_{ij} \sim \mathcal{D}$. In the scope of this thesis, assume every random matrix to be homogenously distributed. Otherwise, if $\D$ is an implicit distribution, then $P$ is a matrix whose entries are determined by the algorithm imposed by $\D$.
\end{definition}

\noindent Sometimes, we want our matrix to have complex entries. We notate this by specifying $\F = \Cc$.

\begin{remark}[Complex Entries]
To say that a random matrix is explicitly $\D$-distributed over $\Cc$ would mean that its entries take the form $a + bi$ where $a,b \sim \D$ are random variables. In other words, if we allow the matrix to have complex entries by setting $\F = \Cc$, then we must sample the real and imaginary component as $\D$-distributed i.i.d. random variables.
\end{remark}

\medskip
\noindent Below, we can see code on how to generate a standard normal random matrix using the $\textbf{RMAT}$ package.
\begin{code}[Standard Normal Matrix]
Let $\mathcal{D} = \Normal(0,1)$. We can generate $P \sim \D$, a $4 \times 4$ standard normal matrix, as such:
\end{code}

\begin{lstlisting}[language=R]
library(RMAT)
P <- RM_norm(N = 4, mean = 0, sd = 1)
# Outputs the following
P
           [,1]       [,2]       [,3]        [,4]
[1,]  0.1058257 -1.0835598 -0.7031727  1.01608625
[2,] -0.2170453  1.8206070 -0.4539230  0.06828296
[3,]  1.3002145  0.1254992 -0.5214005 -0.61516174
[4,] -1.0398587  0.1975445 -0.8511950  0.86366082
\end{lstlisting}

%****************************************************************************************%
\minititle{Summary Table of $\D$-Distributions}
\begin{center}
  \Ddisttable
\end{center}
%****************************************************************************************%
\newpage

%=========================================================================================
%=========================================================================================

\section{The Crew: Ensembles}

With a random matrix well defined, we may now motivate one of the most important ideas - the random matrix ensemble. One common theme in this thesis will be that random matrices on their own provide little information. When we consider them at the ensemble level, we start to obtain more fruitful results. Without further ado, we motivate the random matrix ensemble.

\begin{definition}[Random Matrix Ensemble]
A $\D$-distributed random matrix ensemble $\Ens$ over $\F^{N \times N}$ of size $K$ is defined as a set of $\D$-distributed random matrices $\Ens = \{P_i \sim \mathcal{D} \mid P_i \in \F^{N \times N}\}_{i = 1}^K$. In simple words, it is simply a collection of $K$ iterations of a specified class of random matrix.
\end{definition}

\medskip
\noindent So, for example, we could compute a simple ensemble of matrices as follows.
\begin{code}[Standard Normal Hermitian Ensemble]
Let $\mathcal{D} = \Normal(0,1)^{\dagger}$. We can generate $\Ens \sim \D$ over $\Cc$, an ensemble of $4 \times 4$ complex Hermitian standard normal matrices of size 10 as such:
\end{code}

\begin{lstlisting}[language=R]
library(RMAT)
# By default, mean = 0 and sd = 1.
ensemble <- RME_norm(N = 4, cplx = TRUE, herm = TRUE, size = 10)
# Outputs the following
ensemble
...
[[10]]
                  [,1]              [,2]              [,3]
[1,] -0.59931+1.24286i  1.29457+0.66058i  0.83539-0.16662i
[2,]  1.29457-0.66058i  0.78841+0.09818i -1.16592+1.14666i
[3,]  0.83539+0.16662i -1.16592-1.14666i -0.51256+0.17750i
\end{lstlisting}

With this in mind, we will gloss over, characterize, and briefly discuss a few special recurring ensembles in this thesis.

%=========================================================================================

\subsection{Erdos-Renyi $p$-Ensembles}

The Hermite $\beta$-ensembles are a normal-like class of random matrices. Now, we will veer away from the normal distribution as a whole and switch to a different class of matrices: stochastic matrices. Stochastic matrices, in short, are matrices that represent Markov Chains (see A.x). We can also think of them as the matrix representation of a specific setup of a walk on a random graph.

A particular class of random graphs that we will consider are the Erdos-Renyi random graphs. Essentially, these are graphs whose vertices are connected with a uniform probability $p$. We can interpret this as saying an Erdos-Renyi graph is a simple random walk on a graph with parameterized sparsity (given by $p$). Without further ado, we motivate the Erdos-Renyi graph:

\begin{definition}[Erdos-Renyi Graph]
An Erdos-Renyi graph is a graph $G = (V,E)$ with a set of vertices $V = \{\oneto[N]\}$ and edges $E = \mathds{1}_{i,j \in V} \sim \Bern(p_{ij})$. It is homogenous if $p_{ij} = p$ is fixed for all $i, j$.
\end{definition}

Essentially, an Erdos-Renyi graph is a graph whose 'connectedness' is parameterized by a probability $p$ (assuming it's homogenous, which this document will unless otherwise noted). As $p \to 0$, we say that graph becomes more sparse; analogously, as $p \to 1$ the graph becomes more connected.\newline
\indent Recall from probability theory that a sum of i.i.d Bernoulli random variables is a Binomial variable. As such, we may alternatively say that the degree of each vertex $v$ is distributed as $deg(v) \sim Bin(N,p)$ where $N$ is the number of vertices. This makes simulating the graphs much easier.

\begin{code}[Erdos-Renyi p = 0.5 Ensemble]
Let $\mathcal{D} = \text{ER}(p = 0.5)$. We can generate $\Ens \sim \D$, an ensemble of $4 \times 4$ Erdos-Renyi matrices ($p = 0.5$) of size 10 as such:
\end{code}

\begin{lstlisting}[language=R]
library(RMAT)
ensemble <- RME_erdos(N = 4, p = 0.5, size = 10)
# Outputs the following
ensemble
...
[[10]]
          [,1]      [,2]      [,3]     [,4]
[1,] 0.0000000 0.1729581 0.8270419 0.000000
[2,] 0.0000000 0.0000000 1.0000000 0.000000
[3,] 0.2557890 0.3766740 0.0000000 0.367537
[4,] 0.2151029 0.3929580 0.3919391 0.000000
\end{lstlisting}

%=========================================================================================

\subsection{Hermite $\beta$-Ensembles}

The Hermite $\beta$-Ensembles will be one of the primary ensembles discussed in this thesis. This ensemble will be characterized, motivated, and defined more thoroughly in $\textbf{Chapter 4}$. However, we will give a brief introduction to the ensemble.

At a practical level, all one would need to know is that the matrices are generated in accordance to an algorithm found in the appendix (Algorithm B.x) and in Chapter 4.

%=========================================================================================

%\section{Analytical Results}


%If you feel it necessary to include an appendix, it goes here.
    \appendix
      \chapter{The First Appendix}
      \chapter{The Second Appendix, for Fun}


%This is where endnotes are supposed to go, if you have them.
%I have no idea how endnotes work with LaTeX.

  \backmatter % backmatter makes the index and bibliography appear properly in the t.o.c...

% if you're using bibtex, the next line forces every entry in the bibtex file to be included
% in your bibliography, regardless of whether or not you've cited it in the thesis.
    \nocite{*}

% Rename my bibliography to be called "Works Cited" and not "References" or ``Bibliography''
% \renewcommand{\bibname}{Works Cited}

%    \bibliographystyle{bsts/mla-good} % there are a variety of styles available; 
%  \bibliographystyle{plainnat}
% replace ``plainnat'' with the style of choice. You can refer to files in the bsts or APA 
% subfolder, e.g. 
 \bibliographystyle{APA/apa-good}  % or
 \bibliography{thesis}
 % Comment the above two lines and uncomment the next line to use biblatex-chicago.
 %\printbibliography[heading=bibintoc]

% Finally, an index would go here... but it is also optional.
\end{document}
