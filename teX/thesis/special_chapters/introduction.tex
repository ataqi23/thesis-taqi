%The \introduction command is provided as a convenience.
%if you want special chapter formatting, you'll probably want to avoid using it altogether

  \chapter*{Introduction}
       \addcontentsline{toc}{chapter}{Introduction}
\chaptermark{Introduction}
\markboth{Introduction}{Introduction}
	
	% The three lines above are to make sure that the headers are right, that the intro gets included in the table of contents, and that it doesn't get numbered 1 so that chapter one is 1.
	
What are *spectral statistics*? Do they have to do with rainbows? Sceptres? No, they don't, but they're almost as colorful and regal. *Spectral statistics*, aptly named so, borrows from the spectral-like patterns observed in statistical physics - whether it may be atomic spectra or other quantum mechanical phenomena. The borrowing is loose and not literal, but still somewhat well founded. 

Random matrix theory - as a field - was developed extensively in the 1930s by Wigner, a nuclear physicist. He found connections between the deterministic properties of atomic nuclei and their random and stochastic behaviors. The link? -- Random matrices.   




