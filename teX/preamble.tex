%=================
%     PACKAGES
%=================
\usepackage{graphicx,latexsym} 
\usepackage{amssymb,amsthm,amsmath}
\usepackage{physics}
\usepackage{longtable,booktabs,setspace}
\usepackage[hyphens]{url}
\usepackage{rotating}
\usepackage{natbib}
\usepackage{dsfont} % Boldface 1 for indicator function
\usepackage{array} % For the analytical eigenvector proof (block matrices)
% \usepackage{times} % other fonts are available like times, bookman, charter, palatino

% For code snippet environments
\usepackage{listings} 
\usepackage[svgnames]{xcolor}

% For stitching the code appendix
\usepackage{pdfpages}
%\usepackage{markdown} % For using `script.R` md in code appendix
%\usepackage[hashEnumerators,smartEllipses]{markdown}

%=====================
%     ENVIRONMENTS
%=====================
% Theorems
\newtheorem{theorem}{Theorem}[section]
\setlength{\parskip}{0pt}
\setlength\parindent{24pt}

% Defintions
\newtheorem{definition}{Definition}[section]
\setlength{\parskip}{0pt}
\setlength\parindent{24pt}

% Examples
\newtheorem{example}{Example}[section]
\setlength{\parskip}{0pt}
\setlength\parindent{24pt}

% Code Examples
\newtheorem{code}{Code Example}[section]
\setlength{\parskip}{0pt}
\setlength\parindent{24pt}

% Notes
\newtheorem{note}{Note}[section]
\setlength{\parskip}{0pt}
\setlength\parindent{24pt}

% Algorithms
\newtheorem{algorithm}{Algorithm}[section]
\setlength{\parskip}{0pt}
\setlength\parindent{24pt}
%\newtheorem*{algorithm}{Algorithm}%[section]

% Remarks
\newtheorem{remark}{Remark}[section]
\setlength{\parskip}{0pt}
\setlength\parindent{24pt}

% R Code Chunks
%code snippet environments V1

\lstset{language=R,
    basicstyle=\small\ttfamily,
    stringstyle=\color{DarkGreen},
    otherkeywords={0,1,2,3,4,5,6,7,8,9},
    morekeywords={TRUE,FALSE},
    deletekeywords={data,frame,length,as,character},
    keywordstyle=\color{blue},
    commentstyle=\color{DarkGreen},
}

\xdefinecolor{gray}{rgb}{0.4,0.4,0.4}
\xdefinecolor{blue}{RGB}{58,95,205}% R's royalblue3; #3A5FCD

%==================
%     COMMANDS
%==================
% Latin Letters (bb)
\newcommand{\Cc}{\mathbb{C}} % Complex Numbers
\newcommand{\R}{\mathbb{R}} % Reals
\newcommand{\N}{\mathbb{N}} % Naturals
\newcommand{\F}{\mathbb{F}} % Field

% Latin Letters (cal)
\newcommand{\B}{\mathcal{B}} % Batch
\newcommand{\Rseq}{\mathcal{R}} % Ratio-Sequence
\newcommand{\Seq}{\mathcal{S}} % Sequence
\renewcommand{\S}{\mathbb{S}} % Spectrum
\newcommand{\Ens}{\mathcal{E}} % Ensemble
\newcommand{\D}{\mathcal{D}} % Distribution

% Greek letters
\renewcommand{\epsilon}{\varepsilon}
\newcommand{\ep}{\epsilon}
\renewcommand{\d}{\delta}
\renewcommand{\b}{\beta}

% Probability & Probability Distributions
\newcommand{\Prb}{\text{P}}
\newcommand{\E}{\mathbb{E}}
\newcommand{\Var}{\text{Var}}
\newcommand{\Unif}{\text{Unif}}
\newcommand{\Bern}{\text{Bern}}
\newcommand{\Bin}{\text{Bin}}
\newcommand{\Normal}{\mathcal{N}}

% Math macros
\newcommand{\oneto}[1][n]{1,\dots,#1} % 1,...,n
\newcommand{\sumi}[1][n]{\sum_{i = 1}^{#1}} % sum from i = 1 to n
\newcommand{\seq}[2][n]{{{#2}_0,{#2}_1,\dots,{#2}_{#1}}} % x_1,...,x_n

% Words
\newcommand{\where}{\text{ where }}
\newcommand{\for}{\text{ for }}
\newcommand{\given}{\text{ given }}
\renewcommand{\and}{\text{ and }}

% Other
\newcommand{\ra}{\rightarrow}